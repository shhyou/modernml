令 $W$ 代表整個相關文件 $D'$ 的單字集,包含所有 $unigram$ 至 $4-gram$。則
\textproc{GenerateKeyword} 函式的演算法如下:

\begin{algorithmic}
%  \Comment{$D'\in D$ 是文件子集中相關的部份}
  \Function{GenerateKeyword}{$D'$, \textit{limit}}
    \State \textit{counts} $\gets$ $\forall w\in W,\,\left(w\mapsto|\,\{d\in D':w\in d\}\,|\right)$
    \For{$i=4$ down to $1$}
      \State $W_i$ $\gets$ $W$ 中的 $i-\text{gram}$
      \State Sort $W_i$ by $\mathit{counts}(w)\times {\mathsf{IDF}_w}^2$
      \State $W_i$ $\gets$ $W_i[1..\mathit{limit}]$
      \For{any $w\in W_i$}
        \For{any non-trivial substring $v$ of $w$}
          \State \textit{counts}$(v)$ $\gets$ $\mathit{counts}(v)-\mathit{counts}(w)$
        \EndFor
      \EndFor
    \EndFor
    \State\Return top \textit{limit} ranking of $W$, sorted by
    \State   $\;\;\;\;\;\;\;\mathit{counts}(w)\times{\mathsf{IDF}_w}^2\times i$ for an $i$-gram $w$
  \EndFunction

  在排序過程中,我們透過 \textsf{IDF} 將出現頻率過高又不重要的字排除。在最終的排序中,
  我們將一個單字是幾個字組成也考慮進去,將叫長的字排到前面。把子字串的出現頻率扣掉
  原因如當輸入有 \texttt{content based image retrieval} 等較長的專有名詞時,連帶
  其子字串出現頻率也被拉高,但本身卻是無意義的字。若其子字串原先就有意義,則在此之外
  必須有其它不屬於 3-gram、4-gram 的出現次數。為了避免出現次數被扣到 0,每次我們扣掉
  出現頻率時,僅取 $W_i$ 前 \textit{limit} 個字串來扣。
\end{algorithmic}
