使用者可以在網頁上輸入欲搜尋的關鍵字(或者點選好手氣,由系統隨機挑選關鍵字),送出後在網頁上顯示名詞搜索和流程排序的結果。顯示的結果分成三個部分:
\begin{itemize}
	\item Terms(名詞搜索結果)
		畫面上會顯示至多50個相關的重要名詞,點擊每個名詞標籤後會在上方顯示該名詞的解釋,名詞解釋是從Google搜尋中獲得跟維基相關頁面中,第一個維基頁面內的第一段(\texttt{p} 標籤)內的第一句話取得。另外也有提供使用者自行搜尋其他名詞的欄位
	\item Table of Contents(流程排序結果)
		由上至下顯示流程排序結果,點擊每個標籤會在右方的Relative Link中列出有哪些書或線上課程中有出現這個主題。
	\item Relative Link
		點擊每個標籤會展開內容,裡面會列出書或課程內跟從Table of Contents點擊的主題相關的標題,以及網頁連結。
\end{itemize}

前端用到jQuery、Materialize,後端網頁伺服器是Python Tornado,並且主要的網頁伺服器跟名詞解釋的伺服器是分開的,名詞解釋的部分是另一個Tornado websocket伺服器。