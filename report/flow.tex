% IDF cosine
我們想要幫所有的$s_{i,j}$做clustering和ranking,
建出合理的學習流程。
\subsubsection{Clustering with word vector}

可以先定義兩個$s_{i,j}$的距離,
每個$s_{i,j}$都是一個句子,可以拆成很多word,
然後建構出word vector $v$(像是VSM中的vector),
對於每一項$word$,利用$Count(word)*idf(word)$,
來當作vector該項的數值,最後把vector normalize之後,
就可以用來計算兩個$s_{i,j}$的相似度。

接著就可以$O(N^{2})$掃過所有個$s_{i,j}$,
對於兩個$s_{i,j}$的vector $v_1,v_2$,
如果$\cos(v_1,v_2) > {THRESHOLD}$就將兩個$s_{i,j}$連邊,
最後我們取大小大於${GROUP\;SIZE}$的connected components,
來當作我們的cluster完的groups。

\subsubsection{利用bucket加速}
% bucket 加速 grouping
因為對於不少關鍵字,
搜尋出來的$s_{i,j}$數量級約莫在$10^4$左右,
因此$O(N^{2})$的算法速度會稍嫌慢(尤其我們使用python),
我們利用有出現的word先做${bucket}$,
如果$s_{i,j}$有出現word $w$,那就會被丟到$bucket_w$中,
建邊的動作只在每個bucket中做,
如果平均每個bucket有$m$個$s_{i,j}$,
每個$s_{i,j}$出現在$|s_{i,j}|$個bucket中,
這樣我們計算的複雜度會變成$O(m^{2}\times \frac{N\times n}{m}) = O(mnN)$,
統計上會發現平均$|s_{i,j}| = 5$,
平均$m = 10$,因此大部份時候,算法表現和$O(N)$差不多。

\noindent \paragraph{正確性}
如果$\cos(v1,v2)$能超過$TRESHOLD$,
表示兩個vector至少有一個字相同(不然$\cos$會為零),
如果兩個$s_{i,j}$有至少一個字重複,
就至少會屬於某個$bucket$,
所以他們連邊的計算一定不會少,
因此加速之後結果依然一樣。

\subsubsection{Group Ranking}
% ranking 排法
對於每一個$s_{i,j}$,
我們可以利用其原本在$i/|d_i|$來當作其在$d_i$中的rank,
每個group將其擁有的所有$s_{i,j}$的rank做平均當作自己的rank,
這樣就可以對所有搜尋出來的group做排序。

但會發現有些$s_{i,j}$的字只在某本書大量出現,
而且章節也很前面,所以會被排序到前面,
但卻不是重要的內容,
所以我們將group中有幾個「不同」的$d_i$,也考慮進去,
最後group的rank我們定義為
$\frac{Sum(Rank(s_{i,j}))}{len(group)\times |\{d_i | \exists s_{i,j} \in group\}|}$,
這樣就可以penalize出現過少不同$d_i$的group。

\subsubsection{Buzzword 過濾}
% buzzwords 的過濾
Clustering過程中會發現,
很多$s_{i,j}$會因為一些教科書、課程常出現的buzzwords
而被連cluster在一起,
像是Introduction、Chapter、Midterm等等,
我們可以用bucket解決這個問題,
透過直接忽略buzzwords的bucket,
就可以阻止$s_{i,j}$利用buzzword被cluster在一起。
