\documentclass[twocolumn]{article}

% Math Tpyings
\usepackage{amsmath}
\usepackage{amsfonts}
\usepackage{amssymb}
\usepackage[margin=2cm]{geometry}

% pseudo code
%\usepackage{algorithmicx}
\usepackage{algpseudocode}

%% packages & settings for xeCJK
\usepackage{setspace}
\usepackage{fontspec}
\usepackage{xeCJK}
\setCJKmainfont[AutoFakeBold=4,AutoFakeSlant=.3]{微軟正黑體}
\XeTeXlinebreaklocale "zh"
\XeTeXlinebreakskip = 0pt plus 1pt

\AtBeginDocument{\fontsize{9}{9}\selectfont}

\setlength{\parindent}{12pt}
\setlength{\parskip}{2pt}
\linespread{1.3}

\usepackage{graphicx}

\newcount\aucount
\newcount\originalaucount
\newdimen\auwidth
\auwidth=\textwidth
\newdimen\auskip
\newcount\auskipcount
\newdimen\auskip
\global\auskip=1pc
\newdimen\allauboxes
\allauboxes=\auwidth
\newtoks\addauthors
\newcount\addauflag
\global\addauflag=0 %Haven't shown additional authors yet

\newtoks\subtitletext
\gdef\subtitle#1{\subtitletext={#1}}

\gdef\additionalauthors#1{\addauthors={#1}}

\gdef\numberofauthors#1{\global\aucount=#1
\ifnum\aucount>3\global\originalaucount=\aucount \global\aucount=3\fi %g}  % 3 OK - Gerry March 2007
\global\auskipcount=\aucount\global\advance\auskipcount by 1
\global\multiply\auskipcount by 2
\global\multiply\auskip by \auskipcount
\global\advance\auwidth by -\auskip
\global\divide\auwidth by \aucount}

% \and was modified to count the number of authors.  GKMT 12 Aug 1999
\def\alignauthor{%                  % \begin{tabular}
\end{tabular}%
  \begin{tabular}[t]{p{\auwidth}}\centering}%


%  *** NOTE *** NOTE *** NOTE *** NOTE ***
%  If you have 'font problems' then you may need
%  to change these, e.g. 'arialb' instead of "arialbd".
%  Gerry Murray 11/11/1999
%  *** OR ** comment out block A and activate block B or vice versa.
% **********************************************
%
%  -- Start of block A -- (Type 1 or Truetype fonts)
%\newfont{\secfnt}{timesbd at 12pt} % was timenrb originally - now is timesbd
%\newfont{\secit}{timesbi at 12pt}   %13 Jan 00 gkmt
%\newfont{\subsecfnt}{timesi at 11pt} % was timenrri originally - now is timesi
%\newfont{\subsecit}{timesbi at 11pt} % 13 Jan 00 gkmt -- was times changed to timesbi gm 2/4/2000
%                         % because "normal" is italic, "italic" is Roman
%\newfont{\ttlfnt}{arialbd at 18pt} % was arialb originally - now is arialbd
%\newfont{\ttlit}{arialbi at 18pt}    % 13 Jan 00 gkmt
%\newfont{\subttlfnt}{arial at 14pt} % was arialr originally - now is arial
%\newfont{\subttlit}{ariali at 14pt} % 13 Jan 00 gkmt
%\newfont{\subttlbf}{arialbd at 14pt}  % 13 Jan 00 gkmt
%\newfont{\aufnt}{arial at 12pt} % was arialr originally - now is arial
%\newfont{\auit}{ariali at 12pt} % 13 Jan 00 gkmt
%\newfont{\affaddr}{arial at 10pt} % was arialr originally - now is arial
%\newfont{\affaddrit}{ariali at 10pt} %13 Jan 00 gkmt
%\newfont{\eaddfnt}{arial at 12pt} % was arialr originally - now is arial
%\newfont{\ixpt}{times at 9pt} % was timenrr originally - now is times
%\newfont{\confname}{timesi at 8pt} % was timenrri - now is timesi
%\newfont{\crnotice}{times at 8pt} % was timenrr originally - now is times
%\newfont{\ninept}{times at 9pt} % was timenrr originally - now is times
% *********************************************
%  -- End of block A --
%
%
% -- Start of block B -- UPDATED FONT NAMES
% *********************************************
% Gerry Murray 11/30/2006
% *********************************************
\newfont{\secfnt}{ptmb8t at 12pt}
\newfont{\secit}{ptmbi8t at 12pt}    %13 Jan 00 gkmt
\newfont{\subsecfnt}{ptmri8t at 11pt}
\newfont{\subsecit}{ptmbi8t at 11pt}  % 
\newfont{\ttlfnt}{phvb8t at 18pt}
\newfont{\ttlit}{phvbo8t at 18pt}    % GM 2/4/2000
\newfont{\subttlfnt}{phvr8t at 14pt}
\newfont{\subttlit}{phvro8t at 14pt} % GM 2/4/2000
\newfont{\subttlbf}{phvb8t at 14pt}  % 13 Jan 00 gkmt
\newfont{\aufnt}{phvr8t at 12pt}
\newfont{\auit}{phvro8t at 12pt}     % GM 2/4/2000
\newfont{\affaddr}{phvr8t at 10pt}
\newfont{\affaddrit}{phvro8t at 10pt} % GM 2/4/2000
\newfont{\eaddfnt}{phvr8t at 12pt}
\newfont{\ixpt}{ptmr8t at 9pt}
\newfont{\confname}{ptmri8t at 8pt}
\newfont{\crnotice}{ptmr8t at 8pt}
\newfont{\ninept}{ptmr8t at 9pt}
% +++++++++++++++++++++++++++++++++++++++++++++
% -- End of block B --

%\def\email#1{{{\eaddfnt{\vskip 4pt#1}}}}
% If we have an email, inside a "shared affiliation" then we need the following instead
\def\email#1{{{\eaddfnt{\par #1}}}}       % revised  - GM - 11/30/2006

\def\addauthorsection{\ifnum\originalaucount>6  % was 3 - Gerry March 2007
    \section{Additional Authors}\the\addauthors
  \fi}

\newcount\savesection
\newcount\sectioncntr
\global\sectioncntr=1

\setcounter{secnumdepth}{3}


\pagestyle{empty}

\begin{document}

%\date{} % Remove date
\title{卍煞氣der新世紀高速學習輔助系統卐}

\numberofauthors{3}
\author{
\alignauthor 夏誌陽 \\
       \affaddr{b00901166}
% 2nd. author
\alignauthor 游書泓 \\
       \affaddr{b00902107}
% 3rd. author
\alignauthor 陳亮傑\\
       \affaddr{b01902112}
\and
}
\date{}
\maketitle
\thispagestyle{empty}

\section{簡介}
學習一個新領域時,搜尋結果往往出現各式各樣的專有名詞。面對如此繁雜的
名詞,使用者只能慢慢查詢每個名詞的意義,或從許多書中挑一本來讀。然而
這是相對低效率的學習方式。本系統專書目錄、開放式課程課綱等資料中整理
出重要關鍵字以及推薦學習順序,以輔助使用者學習時快速進入狀況。

\section{系統設計}
本系統的架構如圖片。系統總共由以下元素組成:

\begin{itemize}
  \item 文件 Corpus $D$。其中一份文件 $d_i\in D$ 代表一本書或是一個開放式課程。
  \item 對每份文件 $d_i\in D$ 是一個序列 $d_i = (s_{i,1},\dots,s_{i,m_i})$,
        代表書的目錄或是開放式課程的課表。$s_{i,j}$ 是個字串,為目錄中的
        一個 section 或課程進度表的一個項目。
  \item \textproc{Search} 函數。給定一串搜尋關鍵字 $k$,\textproc{Search}$(k)$
        是所有包含這些關鍵字的文件的集合。
  \item \textproc{GenerateKeyword} 函數。給定文件子集 $D'\subset D$,
        \textproc{GenerateKeyword}$(D')$ 會回傳是文件子集 $D'$ 中的重要關鍵字,
        並會依據重要性做 ranking。
  \item \textproc{GenerateFlow} 函數。給定文件子集 $D'\subset D$,
        \textproc{GenerateFlow}$(D')$ 是將 $D'$ 中文件的目錄或課表適當地 cluster
        與 ranking 形成的整合目錄回傳。
\end{itemize}


\subsection{資訊擷取}
我們的資料分別從 Oreilly.com、Apress.com 以及 MIT open course 中擷取下來,
包含書籍的目錄以及開放式課程的課程大綱。最後蒐集資料有 Apress 5419 筆、
OReilly 1773 筆、MIT Open Course 1522 筆,共計 8714 筆資料。

\subsection{名詞搜索}
% 每個 n-gram 前 limit 個重要 要, n-gram 扣掉 substring, n-gram 由上往下
% IDF 把不好的 keyword 濾掉
令 $W$ 代表整個相關文件 $D'$ 的單字集,包含所有 $unigram$ 至 $4-gram$。則
\textproc{GenerateKeyword} 函式的演算法如下:

\begin{algorithmic}
%  \Comment{$D'\in D$ 是文件子集中相關的部份}
  \Function{GenerateKeyword}{$D'$, \textit{limit}}
    \State \textit{counts} $\gets$ $\forall w\in W,\,\left(w\mapsto|\,\{d\in D':w\in d\}\,|\right)$
    \For{$i=4$ down to $1$}
      \State $W_i$ $\gets$ $W$ 中的 $i-\text{gram}$
      \State Sort $W_i$ by $\mathit{counts}(w)\times {\mathsf{IDF}_w}^2$
      \State $W_i$ $\gets$ $W_i[1..\mathit{limit}]$
      \For{any $w\in W_i$}
        \For{any non-trivial substring $v$ of $w$}
          \State \textit{counts}$(v)$ $\gets$ $\mathit{counts}(v)-\mathit{counts}(w)$
        \EndFor
      \EndFor
    \EndFor
    \State\Return top \textit{limit} ranking of $W$, sorted by
    \State   $\;\;\;\;\;\;\;\mathit{counts}(w)\times{\mathsf{IDF}_w}^2\times i$ for an $i$-gram $w$
  \EndFunction

  在排序過程中,我們透過 \textsf{IDF} 將出現頻率過高又不重要的字排除。在最終的排序中,
  我們將一個單字是幾個字組成也考慮進去,將叫長的字排到前面。把子字串的出現頻率扣掉
  原因如當輸入有 \texttt{content based image retrieval} 等較長的專有名詞時,連帶
  其子字串出現頻率也被拉高,但本身卻是無意義的字。若其子字串原先就有意義,則在此之外
  必須有其它不屬於 3-gram、4-gram 的出現次數。為了避免出現次數被扣到 0,每次我們扣掉
  出現頻率時,僅取 $W_i$ 前 \textit{limit} 個字串來扣。
\end{algorithmic}


\subsection{流程排序}
% IDF cosine
% bucket 加速 grouping
% ranking 排法
% buzzwords 的過濾
% IDF cosine
我們想要幫所有的$s_{i,j}$做clustering和ranking,
建出合理的學習流程。
\subsubsection{Clustering with word vector}

可以先定義兩個$s_{i,j}$的距離,
每個$s_{i,j}$都是一個句子,可以拆成很多word(實驗中包含 unigram 及 bigram),
然後建構出word vector $v$(像是VSM中的vector),
對於每一項 word $w$,利用$\mathsf{TF}_w\times\mathsf{IDF}_w$
來當作vector該項的數值(其中 $\mathsf{TF}_w$ 僅計算 $w$ 出現在 $s_{i,j}$
中的次數),最後把vector normalize之後,
就可以用來計算兩個$s_{i,j}$的相似度。

接著就可以$O(N^{2})$掃過所有個$s_{i,j}$,
對於兩個$s_{i,j}$的vector $v_1,v_2$,
如果$\cos(v_1,v_2) > \text{\textproc{Threshold}}$就將兩個$s_{i,j}$連邊,
最後我們取大小大於\textproc{Group Size}的connected components,
來當作我們的cluster完的groups。

\subsubsection{利用bucket加速}
% bucket 加速 grouping
因為對於不少關鍵字,
搜尋出來的$s_{i,j}$數量級約莫在$10^4$左右,
因此$O(N^{2})$的算法速度會稍嫌慢(尤其我們使用python),
我們利用有出現的word先做$\mathit{bucket}$,
如果$s_{i,j}$有出現word $w$,那就會被丟到$bucket_w$中,
建邊的動作只在每個bucket中做,
如果平均每個bucket有$m$個$s_{i,j}$,
每個$s_{i,j}$出現在$|s_{i,j}|$個bucket中,
這樣我們計算的複雜度會變成$O(m^2\times \frac{N\times n}{m}) = O(mnN)$,
統計上會發現平均$|s_{i,j}| = 5$,
平均$m = 10$,因此大部份時候,算法表現和$O(N)$差不多。

\noindent \paragraph{正確性}
如果$\cos(v1,v2)$能超過 \textproc{Threshold},
表示兩個vector至少有一個字相同(不然$\cos$會為零),
如果兩個$s_{i,j}$有至少一個字重複,
就至少會屬於某個$\mathit{bucket}$,
所以他們連邊的計算一定不會少,
因此加速之後結果依然一樣。

\subsubsection{Group Ranking}
% ranking 排法
對於每一個$s_{i,j}$,
我們可以利用其原本在$i/|d_i|$來當作其在$d_i$中的rank,
每個group將其擁有的所有$s_{i,j}$的rank做平均當作自己的rank,
這樣就可以對所有搜尋出來的group做排序。

但會發現有些$s_{i,j}$的字只在某本書大量出現,
而且章節也很前面,所以會被排序到前面,
但卻不是重要的內容,
所以我們將group中有幾個「不同」的$d_i$,也考慮進去,
最後group的rank我們定義如下。我們以此penalize出現過少不同$d_i$的group。
\[ \frac
    {\sum \mathit{Rank}(s_{i,j})}
    {|\mathit{group}|\times |\{d_i \,|\, \exists s_{i,j} \in group\}|} \]

\subsubsection{Buzzword 過濾}
% buzzwords 的過濾
Clustering過程中會發現,
很多$s_{i,j}$會因為一些教科書、課程常出現的buzzwords
而被連cluster在一起,
像是Introduction、Chapter、Midterm等等,
我們可以用bucket解決這個問題,
透過直接忽略buzzwords的bucket,
就可以阻止$s_{i,j}$利用buzzword被cluster在一起。


\subsection{使用者介面}
使用者可以在網頁上輸入欲搜尋的關鍵字(或者點選好手氣,由系統隨機挑選關鍵字),送出後在網頁上顯示名詞搜索和流程排序的結果。顯示的結果分成三個部分:
\begin{itemize}
	\item Terms(名詞搜索結果)
		畫面上會顯示至多50個相關的重要名詞,點擊每個名詞標籤後會在上方顯示該名詞的解釋,名詞解釋是從Google搜尋中獲得跟維基相關頁面中,第一個維基頁面內的第一段(\texttt{p} 標籤)內的第一句話取得。另外也有提供使用者自行搜尋其他名詞的欄位
	\item Table of Contents(流程排序結果)
		由上至下顯示流程排序結果,點擊每個標籤會在右方的Relative Link中列出有哪些書或線上課程中有出現這個主題。
	\item Relative Link
		點擊每個標籤會展開內容,裡面會列出書或課程內跟從Table of Contents點擊的主題相關的標題,以及網頁連結。
\end{itemize}

前端用到jQuery、Materialize,後端網頁伺服器是Python Tornado,並且主要的網頁伺服器跟名詞解釋的伺服器是分開的,名詞解釋的部分是另一個Tornado websocket伺服器。

\section{結論}
這系統說真的,很煞氣。

\section{致謝}
老姜好威 感謝老姜

\end{document}
